\documentclass[xcolor=dvipsnames]{beamer} 
\usecolortheme[named=Blue]{structure} 
\usetheme[height=10.5mm]{Rochester} 
\setbeamertemplate{items}[ball] 
\setbeamertemplate{blocks}[rounded][shadow=true] 
\setbeamertemplate{navigation symbols}{} 
\usepackage{bm}
\usepackage{listings}
\usepackage{rotating}
\usepackage{graphicx}
\usepackage{multirow}
\usepackage{hyperref}
\usepackage{textcomp}
\usepackage{upquote}
\usepackage[absolute,overlay]{textpos}
\newenvironment{reference}[2]{%
  \begin{textblock*}{\textwidth}(#1,#2)
      \footnotesize\it\bgroup\color{red!50!black}}{\egroup\end{textblock*}}
%\graphicspath{ {/home/ben/PhD/Armidale_Updates/2014_03_14/Figures/} }
\begin{document}

%%% New plan for this module:
%%% we all use Git on the command line and view GitHub in our web browsers
%%% clone and commit via HTTPS
%%% Workflow plan:
%%%
%%% Create repo in web browser
%%% Authenticate on cmd line
%%% Clone
%%% Add files
%%% Edit
%%% git add
%%% git commit
%%% branch and revert to previous version with check out


%%% Getting on the command line in various OS's

%%% MS Windows -> Start Menue -> Powershell
%%% Mac OS -> Finder -> terminal
%%% GNU+Linux users -> application launcher (will vary depending on GUI) -> terminal 


\begin{frame} %1
% \frametitle{Addressing Common Challenges in Spatial Modeling of Ecologies and Environments}
\begin{center}
\textbf{\huge Open Research with Open Source Software}\\
\end{center}

\begin{figure}
%\includegraphics[width = 0.35\textwidth]{/home/ben/Intro_to_R/GitHub_Slides_Source/Images/GitHub-Mark-120px-plus.png}
\begin{columns}

\begin{column}{3.3cm}
\begin{center}
\begin{figure}
\includegraphics[width = 0.2\textheight]{/home/ben/Open_Access_Week_2015/Open_Science_Talk/Slides/Figures/GNU_700.png}
\end{figure}
\end{center}
\end{column} 

\begin{column}{3.3cm}
\begin{center}
\begin{figure}
\includegraphics[width = 0.2\textheight]{/home/ben/Open_Access_Week_2015/Open_Science_Talk/Slides/Figures/Tux_700h.png}
\end{figure}
\end{center}
\end{column} 

\begin{column}{3.3cm}
\begin{center}
\begin{figure}
\includegraphics[width = 0.2\textheight]{/home/ben/Open_Access_Week_2015/Open_Science_Talk/Slides/Figures/OA_700h.png}
\end{figure}
\end{center}
\end{column} 

\end{columns}
\end{figure}

\small Ben R. Fitzpatrick\\
\tiny PhD Candidate, Statistical Science, Mathematical Sciences School, Queensland University of Technology
\newline
\begin{columns}
\begin{column}{3cm}
\tiny 0000-0003-1916-0939
\end{column}
\begin{column}{3cm}
\tiny github.com/brfitzpatrick/
\end{column}
\begin{column}{3cm}
\tiny @benrfitzpatrick
\end{column}
\end{columns}
\end{frame}

\begin{frame}
\frametitle{Why Use Free and Open Source Software for Research?}
\begin{block}{Ideological Reasons}
free software means users have the four essential freedoms: \begin{enumerate}
 \setcounter{enumi}{-1}
 \item to run the program, 
 \item to study and change the program in source code form,
 \item to redistribute exact copies and
 \item to distribute modified versions. \end{enumerate}
\begin{center} \small- \url{https://www.gnu.org/philosophy/philosophy.html}
\end{center}
\end{block}

\begin{center}
\begin{figure}
\includegraphics[width = 0.3\textheight]{/home/ben/Open_Access_Week_2015/Open_Science_Talk/Slides/Figures/GNU_700.png}
\end{figure}
\end{center}


\end{frame}


\begin{frame}

\begin{block}{Practical Reasons}
Makes your research workflow:
\begin{itemize}
\item cheaper
\item more reproducible 
\newline
\end{itemize}

Makes you:
\begin{itemize}
\item easier to collaborate with 
\item able to work anywhere with a computer and internet connection
\item Part of a scientific community that includes \begin{itemize}
      \item \href{https://training.linuxfoundation.org/why-our-linux-training/training-reviews/linux-foundation-training-prepares-the-international-space-station-for-linux-migration}{International Space Station} 
      \item \href{https://linux.web.cern.ch/linux/scientific.shtml}{CERN}
\end{itemize}
\end{itemize}

\end{block}

\end{frame}

\begin{frame}
\frametitle{Some of My Favourite Free and Open Source Software for Academia}
\begin{tabular}{ll}
Operating System       & \href{https://www.gnu.org/}{GNU} + \href{https://kernel.org/}{Linux} \\
                       & \\
Web Browser            & \href{https://www.mozilla.org}{Firefox} \\
                       & \\
Reference Managers     &  \href{https://www.zotero.org/}{Zotero} \& \href{http://jabref.sourceforge.net/}{JabRef} \\
                       & \\
Document Preparation   & \href{https://www.latex-project.org/}{LaTeX} \\
                       & \\
Writing Code (IDE)     & \href{https://www.gnu.org/software/emacs/}{Emacs} \\
                       & \\
Statistics \& Graphics & \href{https://cran.r-project.org/}{R} \\
              & \\
Geographic Information System & \href{http://www.saga-gis.org/en/index.html}{SAGA} \\
           & \\
Version Control & \href{https://git-scm.com/}{Git} \\
           & \\
Diff Tool & \href{http://meldmerge.org/}{Meld} \\
           & \\
Document Conversion & Pandoc \\
\end{tabular}

\end{frame}

\begin{frame}
\frametitle{Alternatives}
\begin{tabular}{ll}
Document Preparation & \href{https://www.libreoffice.org/}{Libre Office} \& \href{https://www.openoffice.org/}{Apache Open Office} \\
& \\
Document Conversion & \href{http://pandoc.org/}{Pandoc} \\
\end{tabular}

\end{frame}


\begin{frame}
\frametitle{Demo}
\begin{enumerate}
\item quick look at Libre Office Writer, Calc and Impress 
\newline
\item reference management with Zotero and or JabRef
\newline
\item a longer look at LaTeX including \begin{enumerate}  
      \item an article with a Journal Style File
      \item a slide show with (Beamer)
      \item a poster 
\newline
\end{enumerate}
\item version control of LaTeX documents
\end{enumerate}

\end{frame}


\begin{frame}
Compiled Slides on FigShare
.tex Source on GitHub

\end{frame}


\begin{frame}
\frametitle{Free and Open Source Software useful for Academia}
\begin{center}
\begin{figure}
\includegraphics[height = 0.9\textheight]{/home/ben/Open_Access_Week_2015/Open_Science_Talk/Slides/Figures/oa_week_fillled_hex_clip.png}
\end{figure}
\end{center}
\end{frame}

\begin{frame}
\frametitle{Code Repository Hosting}
\begin{center}

\begin{figure}
\includegraphics[height = 0.3\textheight]{/home/ben/Open_Access_Week_2015/Open_Science_Talk/Slides/Figures/Atlassian_Bitbucket_Logo.png}
\end{figure}

\begin{figure}
\includegraphics[height = 0.15\textheight]{/home/ben/Open_Access_Week_2015/Open_Science_Talk/Slides/Figures/GitHub.png}
\end{figure}
\end{center}

\end{frame}




\end{document}
