\documentclass[xcolor=dvipsnames]{beamer} 
\usecolortheme[named=Blue]{structure} 
\usetheme[height=10.5mm]{Rochester} 
\setbeamertemplate{items}[ball] 
\setbeamertemplate{blocks}[rounded][shadow=true] 
\setbeamertemplate{navigation symbols}{} 
\usepackage{bm}
\usepackage{listings}
\usepackage{rotating}
\usepackage{graphicx}
\usepackage{multirow}
\usepackage{hyperref}
\usepackage{textcomp}
\usepackage{upquote}
\usepackage[absolute,overlay]{textpos}
\newenvironment{reference}[2]{%
  \begin{textblock*}{\textwidth}(#1,#2)
      \footnotesize\it\bgroup\color{red!50!black}}{\egroup\end{textblock*}}
\begin{document}

\begin{frame} 
\begin{center}
\textbf{\huge Open Research with Open Source Software}\\
\end{center}

\begin{figure}
\begin{columns}

\begin{column}{3.3cm}
\begin{center}
\begin{figure}
\includegraphics[width = 0.2\textheight]{/home/ben/Open_Access_Week_2015/Open_Science_Talk/Slides/Figures/GNU_700.png}
\end{figure}
\end{center}
\end{column} 

\begin{column}{3.3cm}
\begin{center}
\begin{figure}
\includegraphics[width = 0.2\textheight]{/home/ben/Open_Access_Week_2015/Open_Science_Talk/Slides/Figures/Tux_700h.png}
\end{figure}
\end{center}
\end{column} 

\begin{column}{3.3cm}
\begin{center}
\begin{figure}
\includegraphics[width = 0.2\textheight]{/home/ben/Open_Access_Week_2015/Open_Science_Talk/Slides/Figures/OA_700h.png}
\end{figure}
\end{center}
\end{column} 

\end{columns}
\end{figure}

\small Ben R. Fitzpatrick\\
\tiny PhD Candidate, Statistical Science, Mathematical Sciences School, Queensland University of Technology
\newline
\begin{columns}
\begin{column}{3cm}
\tiny 0000-0003-1916-0939
\end{column}
\begin{column}{3cm}
\tiny github.com/brfitzpatrick/
\end{column}
\begin{column}{3cm}
\tiny @benrfitzpatrick
\end{column}
\end{columns}
\end{frame}

\begin{frame}
\frametitle{Why Use Free and Open Source Software for Research?}
\begin{block}{Ideological Reasons}
free software means users have the four essential freedoms: \begin{enumerate}
 \setcounter{enumi}{-1}
 \item to run the program, 
 \item to study and change the program in source code form,
 \item to redistribute exact copies and
 \item to distribute modified versions. \end{enumerate}
\begin{center} \small- \url{https://www.gnu.org/philosophy/philosophy.html}
\end{center}
\end{block}

\begin{center}
\begin{figure}
\includegraphics[width = 0.3\textheight]{/home/ben/Open_Access_Week_2015/Open_Science_Talk/Slides/Figures/GNU_700.png}
\end{figure}
\end{center}
\end{frame}

\begin{frame}
\frametitle{Why Use Free and Open Source Software for Research?}
\begin{block}{Practical Reasons}
Makes your research workflow:
\begin{itemize}
\item cheaper
\item more reproducible 
\newline
\end{itemize}

Makes you:
\begin{itemize}
\item easier to collaborate with 
\item able to work anywhere where there is a computer \& internet connection
\item Part of a scientific community of open source software users that includes \begin{itemize}
      \item \href{https://training.linuxfoundation.org/why-our-linux-training/training-reviews/linux-foundation-training-prepares-the-international-space-station-for-linux-migration}{International Space Station} 
      \item \href{https://linux.web.cern.ch/linux/scientific.shtml}{CERN}
\end{itemize}
\end{itemize}

\end{block}

\end{frame}

\begin{frame}
\frametitle{My Favourite Free and Open Source Software for Academia}
\begin{tabular}{ll}
Operating System              & \href{https://www.gnu.org/}{GNU} + \href{https://kernel.org/}{Linux} \\
                              & \\
Web Browser                   & \href{https://www.mozilla.org}{Firefox} \\
                              & \\
Reference Managers            &  \href{https://www.zotero.org/}{Zotero} \& \href{http://jabref.sourceforge.net/}{JabRef} \\
                              & \\
Document Preparation          & \href{https://www.latex-project.org/}{LaTeX} \\
                              & \\
Writing Code (IDE)            & \href{https://www.gnu.org/software/emacs/}{Emacs} \\
                              & \\
Statistics \& Graphics        & \href{https://cran.r-project.org/}{R} \\
                              & \\
Geographic Information System & \href{http://www.saga-gis.org/en/index.html}{SAGA} \\
                              & \\
Version Control               & \href{https://git-scm.com/}{Git} \\
                              & \\
Diff Tool                     & \href{http://meldmerge.org/}{Meld} \\
                              & \\
Document Conversion           & Pandoc \\
\end{tabular}

\end{frame}

\begin{frame}
\frametitle{Demo}
A quick look at \href{https://www.libreoffice.org/}{Libre Office}
\begin{center}
\begin{figure}
\includegraphics[width = \textwidth]{/home/ben/Open_Access_Week_2015/Open_Science_Talk/Slides/Figures/Libre_Office_Writer.png}
\end{figure}
\end{center}
Libre Office Writer, Calc \& Impress are inspired by MS Word, Excel \& Powerpoint respectively.
\end{frame}

\begin{frame}
\frametitle{Demo}
A quick look reference management with \href{https://www.zotero.org/}{Zotero}.
\begin{center}
\begin{figure}
\includegraphics[width = \textwidth]{/home/ben/Open_Access_Week_2015/Open_Science_Talk/Slides/Figures/Zotero_Screenshot.png}
\end{figure}
\end{center}


\end{frame}

\begin{frame}
\frametitle{Demo}
A longer look at LaTeX including: \begin{enumerate}  
      \item an Article formatted with a Journal Style File
\newline
      \item a Slide Show with LaTeX class `Beamer'
\newline
      \item a Poster also formatted with `Beamer'
\newline
\newline
\end{enumerate}
\href{http://pandoc.org/}{Pandoc} will convert simple .tex files to .docx files reasonably well...
\end{frame}

\begin{frame}
\frametitle{Collaborating \& Tracking Changes}
LaTeX files (.tex) are code files \& so may be version controlled with tradition version control software e.g. \href{https://git-scm.com/}{Git}.
\newline
\newline
Version control of a .tex file allows collaborative editing and change tracking functionality analogous to MS Word style tracked changes and comments along with much more...
\newline
\newline
Code Repository Hosting:
\begin{center}

\begin{figure}
\includegraphics[height = 0.3\textheight]{/home/ben/Open_Access_Week_2015/Open_Science_Talk/Slides/Figures/Atlassian_Bitbucket_Logo.png}
\end{figure}

\begin{figure}
\includegraphics[height = 0.15\textheight]{/home/ben/Open_Access_Week_2015/Open_Science_Talk/Slides/Figures/GitHub.png}
\end{figure}
\end{center}
\end{frame}

\begin{frame}
\frametitle{Thanks for Listening...Any Questions?}
\begin{center}
\begin{figure}
\includegraphics[height = 0.9\textheight]{/home/ben/Open_Access_Week_2015/Open_Science_Talk/Slides/Figures/oa_week_fillled_hex_clip.png}
\end{figure}
\end{center}
\end{frame}

\begin{frame}
\begin{block}{Compiled Slides:}
\small \url{http://dx.doi.org/10.6084/m9.figshare.1579611}
\end{block}

\begin{block}{Slides Source:}
\small \url{https://github.com/brfitzpatrick/OAWeek2015_Slides}
\end{block}
\end{frame}
\end{document}
